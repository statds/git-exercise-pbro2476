\documentclass[12pt, letterpaper, twoside]{article}

\usepackage{amsmath}
\usepackage{booktabs}
\usepackage{amsthm}
\usepackage{graphicx}
\usepackage[margin=1in]{geometry}
\usepackage{hyperref}
\hypersetup{colorlinks = true, linkcolor = blue, citecolor=blue, urlcolor = blue}
\usepackage{natbib}
\usepackage{enumitem}
\usepackage{setspace}
\graphicspath{ {/Users/patrickbrogan/Desktop/R/} }

\usepackage[pagewise]{lineno}
%\linenumbers*[1]
% %% patches to make lineno work better with amsmath
\newcommand*\patchAmsMathEnvironmentForLineno[1]{%
 \expandafter\let\csname old#1\expandafter\endcsname\csname #1\endcsname
 \expandafter\let\csname oldend#1\expandafter\endcsname\csname end#1\endcsname
 \renewenvironment{#1}%
 {\linenomath\csname old#1\endcsname}%
 {\csname oldend#1\endcsname\endlinenomath}}%
\newcommand*\patchBothAmsMathEnvironmentsForLineno[1]{%
 \patchAmsMathEnvironmentForLineno{#1}%
 \patchAmsMathEnvironmentForLineno{#1*}}%
 
 \AtBeginDocument{%
 \patchBothAmsMathEnvironmentsForLineno{equation}%
 \patchBothAmsMathEnvironmentsForLineno{align}%
 \patchBothAmsMathEnvironmentsForLineno{flalign}%
 \patchBothAmsMathEnvironmentsForLineno{alignat}%
 \patchBothAmsMathEnvironmentsForLineno{gather}%
 \patchBothAmsMathEnvironmentsForLineno{multline}%
}

\title{STAT 3494W Paper}
\author{Patrick Brogan\\[1ex]
  Department of Statistics, University of Connecticut\\}
\date{August 2022}

\begin{document}

\maketitle

\begin{abstract}
One-paragraph summary of the entire study – typically no more than
250 words in length (and in many cases it is well shorter than that),
the Abstract provides an overview of the study.
\end{abstract}

\section*{Introduction}
\addcontentsline{toc}{section}{Introduction}
What is the topic and why is it worth studying? – the first major
section of text in the paper, the Introduction commonly describes
the topic under investigation, summarizes or discusses relevant
prior research (for related details, please see the Writing Literature
Reviews section of this website), identifies unresolved issues that
the current research will address, and provides an overview of the
research that is to be described in greater detail in the sections to
follow.

\section*{Methods}
\addcontentsline{toc}{section}{Methods}
What did you do? – a section which details how the research was
performed.  It typically features a description of the participants/
subjects that were involved, the study design, the materials that were
used, and the study procedure.  If there were multiple experiments,
then each experiment may require a separate Methods section.  A rule
of thumb is that the Methods section should be sufficiently detailed
for another researcher to duplicate your research. Inline equations should look like \(e^{i\theta} = \cos(\theta) + i\sin\theta\). Use displaystyle equations like so \begin{equation}
  \label{eq:ks_standard}
  f(y) = \frac{1}{\Gamma(\alpha)\beta^\alpha}\*y^{\alpha-1}e^{-\frac{y}{\beta}}
\end{equation}

\section*{Results}
\addcontentsline{toc}{section}{Results}
What did you find? – a section which describes the data that was
collected and the results of any statistical tests that were performed.
It may also be prefaced by a description of the analysis procedure that
was used. If there were multiple experiments, then each experiment may
require a separate Results section.

\begin{center}
\begin{tabular}{||c c c c||} 
 \hline
 Uniform & Exponential & Normal & Gamma \\ [0.5ex] 
 \hline\hline
 0.3752854 & 0.8941359 & 1.0891278 & 1.8365839 \\ 
 \hline
 0.3429779 & 1.5657469 & 0.6783542 & 0.2391189 \\
 \hline
 0.5412031 & 2.0190394 & -0.2033725 & 1.0632763 \\
 \hline
 0.0224390 & 1.2995342 & 0.2241231 & 0.2357952 \\
 \hline
 0.4368829 & 0.2267861 & -0.3625344 & 3.3227355 \\ [1ex] 
 \hline
\end{tabular}
\end{center}

\begin{figure}[h]
  \centering
  \caption{Table of random numbers generated from different probability distributions}
  \label{fig:Random Variable Table}
\end{figure}

\includegraphics[scale=0.3]{GammaHist}

\begin{figure}[h]
  \centering
  \caption{Histogram of randomly generated numbers from a gamma probability distributions}
  \label{fig:Random Gamma Histogram}
\end{figure}

\section*{Discussion}
What is the significance of your results? – the final major section of
text in the paper.  The Discussion commonly features a summary of the
results that were obtained in the study, describes how those results
address the topic under investigation and/or the issues that the research
was designed to address, and may expand upon the implications of those
findings.  Limitations and directions for future research are also commonly
addressed.

\newpage

\section*{References}
List of articles and any books cited – an alphabetized list of the
sources that are cited in the paper (by last name of the first author of
each source).  Each reference should follow specific APA guidelines
regarding author names, dates, article titles, journal titles, journal
volume numbers, page numbers, book publishers, publisher locations,
websites, and so on (for more information, please see the Citing References
in APA Style page of this website).

\end{document}
